\documentclass{emulateapj}

\shorttitle{Accretion Disks in 2D HL Flow}
\shortauthors{Blondin}

\begin{document}
 
\title{Accretion Disks in Two-Dimensional Hoyle-Lyttleton Flow}

\author{John M. Blondin}
\affil{Department of Physics, North Carolina State University, Raleigh, NC 27695-8202}
\email{John\_Blondin@ncsu.edu}

\begin{abstract}

We investigate the flip-flop instability observed in two dimensional planar hydrodynamic simulations 
of Hoyle-Lyttleton accretion in the case of an accreting object with a radius much smaller
than the nominal accretion radius, as one would expect in astrophysically relevant situations.
Contrary to previous results with larger accretors, accretion from a homogenous
medium onto a small accretor is characterized by a robust, quasi-Keplerian accretion disk. 

\end{abstract}



\section{Introduction}

The accretion-driven X-ray luminosity of many high-mass X-ray binaries is often modeled in terms of the classical
Hoyle-Lyttleton accretion theory (hereafter referred to as HLA). 
\cite{do73} originally proposed the accretion of a companion stellar wind as the source of X-ray luminosity from Cyg X-3.

The HLA problem is parameterized by an object of mass $M$ moving at a speed $V_\infty$ through a uniform 
medium of density $\rho_\infty$.  Using ballistic orbits (neglecting pressure effects), \cite{hl39a} defined an accretion radius given by
\begin{equation}
R_a=\frac{2GM}{V_\infty^2},
\end{equation}
such that gas approaching the star with an impact parameter less than $R_a$ would collide on an accretion line 
behind the star and would be left with insufficient kinetic energy to escape the gravitational potential of the star.  
This model predicts a mass accretion rate given by the mass flux through a circle of radius $R_a$ far upstream:
\begin{equation}
\dot M_{HL} = \pi R_a^2 V_\infty \rho_\infty.
\label{eqn:mdot}
\end{equation}

Given the symmetry of this idealized model and the assumption that the accretion is dictated by the upstream flow,
the angular momentum accretion rate is identically zero: $\dot J_{HL}= 0$.

This estimate of the mass accretion rate was confirmed by two-dimensional, time-independent calculations of axisymmetric accretion flow.
For an upstream Mach number of 2.4, \cite{h71} found
an accretion rate of $0.88\, \dot M_{HL}$.  This is within 3\% of the expected value if one corrects for the finite Mach 
number \citep{bondi52}.  Time-dependent simulations of axisymmetric accretion exhibit steady flow with mass accretion
rates close to this analytic prediction.
Despite this agreement between Equation (\ref{eqn:mdot}) and numerical simulations, it is important to note that the flow does not follow a ballistic
trajectory all the way to an accretion column on the back side of the accretor.
For an ideal gas with a ratio of specific heats of $\gamma=5/3$, axisymmetric accretion flows are characterized by a
leading bow shock \citep{h71} rather than a trailing accretion line as originally envisioned by \cite{hl39a}.  

The magnitude of the angular momentum accretion rate is less certain when one considers the context of wind accretion in a close binary system.  
In the original application of HLA to X-ray binaries, \citet{do73} pointed out that the orbital motion of the accreting neutron
star leads to a relative flow velocity that is no longer radially away from the donor star.  This, combined with the 
decreasing density of the wind from the
donor star due to spherical divergence, leads to a variation of the wind density across the face of the accretion cylinder.  The 
assumption that all matter flowing within the accretion cylinder will be accreted led them to estimate a relatively large value for $\dot J$.
\citet{sl76} accounted for the changes in the accretion radius due to an accelerating wind, but found that the dominant effect was
still the variation of the wind density, predicting a large $\dot J$ in wind-fed X-ray binaries.  In contrast, \citet{dp80} argued that,
following the original argument of momentum balance on a trailing accretion column \citep{hl39a}, any upstream gradients would simply lead
to a displacement of this column that forced $\dot J = 0$.

This scenario of wind-fed accretion in a binary system was investigated numerically by \citet{mis87}.  
They presented two-dimensional simulations of the gas flow in the equatorial
plane of a semi-detached binary system with a mass ratio of unity.  They found that wind-fed accretion can lead to
accretion of significant angular momentum.  Moreover, most of their models exhibited unsteady flow.  In the most violent examples,
an asymmetric bow shock flipped from one side to the other, leading to alternating clockwise and counter-clockwise rotating flows
about the accreting compact star.  






The highest mass accretion rate
occurs in brief flares that last of order 100 orbital periods at the surface of the accretor, or roughly one minute for
canonical X-ray binary parameters.  For reference, the orbital period at the surface of the accretor
is $5\times 10^{-4}\,(R_a/V_\infty)$ for this model.  
These flares are the result of accretion of high-density structures within the accretion disk.
The density pattern shown in Figure \ref{fig:disks}
exhibits a relatively tightly-wound one-armed spiral.  This spiral pattern continually changed throughout the evolution, but the flow
remained nearly Keplerian (within 20\%) out to $10\, R_s$.  The Mach number in this disk varies between one and two.  
Describe time evolution - waves pushing out on bow shock.  Note that the strong spiral shocks in this relatively low-Mach disk lead to 
very effective radial transport of mass and angular momentum and hence a relatively brief residence time of the high-density 
structures in the disk.  

\begin{figure}[!htp]
\begin{center}
\includegraphics[width=3.25in]{f3}
\caption{Density structure in the $\gamma=5/3$ model at randomly selected times.  Each panel has a width of $R_a$ - at this scale
the accreting star is unresolved.  The highest density is shown in red.}
\label{fig:disks}
\end{center}\end{figure}



\section{Discussion}

Under the restriction of adiabatic two dimensional planar flow, gravitational accretion onto a star with $R_s \ll R_a$ does not behave according to 
classical HLA. Rather, accretion in this regime is characterized by a quasi-Keplerian
accretion disk, a mass accretion rate much less than that predicted by HLA, and an angular momentum accretion rate appropriate to
disk accretion.

In this disk mode, the accretion of mass and angular momentum is unrelated to the flux through an accretion cylinder far upstream.
The mass accretion rate in such a disk phase is determined not by the upstream flow, but by the efficiency of 
spiral shocks (or other physical mechanisms not included in our simulations) in transporting angular momentum 
outwards and mass inwards through the disk.  Similarly, the specific angular momentum of the accreted gas is determined solely by
the conditions at the inner edge of the accretion disk.  Furthermore, the rotational direction of the disk is set by the 
morphology of the accretion bow shock (which is dynamically interacting with the disk flow itself),
not by the presence of a net angular momentum flowing through an upstream accretion cylinder. 


\begin{thebibliography}{}

\bibitem[Bondi(1952)]{bondi52}Bondi, H. 1952, MNRAS, 112, 195

\bibitem[Davidson \& Ostriker(1973)]{do73}Davidson, K., \& Ostriker, J. P. 1973, ApJ, 179, 585

\bibitem[Davies \& Pringle(1980)]{dp80}Davies, R. E., \& Pringle, J. E. 1980, MNRAS, 191, 599

\bibitem[Hunt(1971)]{h71}Hunt, R. 1971, MNRAS, 154, 141

\bibitem[Hoyle \& Lyttleton(1939)]{hl39a}Hoyle, F., \& Lyttleton R. A. 1939, Proc, Cambridge Phil. Soc., 35, 405

\bibitem[Matsuda et al.(1987)]{mis87}Matsuda, T., Inoue, M., \& Sawada, K. 1987, MNRAS 226, 785

\bibitem[Shapiro \& Lightman(1976)]{sl76}Shapiro, S. L., \& Lightman, A. P. 1976, ApJ, 204, 555

\end{thebibliography}
\clearpage

\end{document}
